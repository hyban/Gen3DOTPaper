%%%%%%%%%%%%%%%%%%%%%%%%%%%%%%%%%%%%%%%%%%%%%%%%%%%%%%%
%                File: OpEx_temp.tex                  %
%                Date: April 30, 2014                 %
%                                                     %
%           LaTeX template file for use with          %
%           OSA's journal Optics Express              %
%                                                     %
%  send comments to Theresa Miller, tmiller@osa.org   %
%                                                     %
% This file requires style file, opex3.sty, under     %
%              the LaTeX article class                %
%                                                     %
%   \documentclass[10pt,letterpaper]{article}         %
%   \usepackage{opex3}                                %
%                                                     %
%                                                     %
%       (c) 2014 Optical Society of America           %
%%%%%%%%%%%%%%%%%%%%%%%%%%%%%%%%%%%%%%%%%%%%%%%%%%%%%%%

%%%%%%%%%%%%%%%%%%%%%%% preamble %%%%%%%%%%%%%%%%%%%%%%%%%%%
\documentclass[10pt,letterpaper]{article}
\usepackage{opex3}
\usepackage{color}

%%%%%%%%%%%%%%%%%%%%%%% begin %%%%%%%%%%%%%%%%%%%%%%%%%%%%%%
\begin{document}

\title{Working Title}
\author{Author 1$^1$ and Author 2$^{2,*}$}

\address{$^1$Department of Physics and Astronomy, University of Pennsylvania, Philadelphia, PA 19104, USA\\
$^2$Department of Computer Science, University College London, London, UK}

\email{$^*$yodh@physics.upenn.edu} %% email address is required


%%%%%%%%%%%%%%%%%%% abstract and OCIS codes %%%%%%%%%%%%%%%%
%% [use \begin{abstract*}...\end{abstract*} if exempt from copyright]

\begin{abstract}
We are developing a 3rd generation breast imaging device based on Diffuse Optical Tomography (DOT).
The device improves on our previous experience in diffuse optical instruments in several ways.
Here we describe the implementation and optimization of various features in our 3rd generation device
and present preliminary data from the instrument. New features include very large source-detector pairs,
multi-spectral imaging, and simultaneous frequency domain and continuous-wave data acquisition through
heterodyne detection.
\end{abstract}

\ocis{(000.0000) General.} % REPLACE WITH CORRECT OCIS CODES FOR YOUR ARTICLE, MINIMUM OF TWO; Avoid using the OCIS codes for “General” or “General science” whenever possible.
%For a complete list of OCIS codes, visit: http://www.opticsinfobase.org/submit/ocis/

%%%%%%%%%%%%%%%%%%%%%%% References %%%%%%%%%%%%%%%%%%%%%%%%%
\begin{thebibliography}{99}
\bibitem{bib1}P. J. Harshman, T. K. Gustafson, and P. Kelley, "Title of paper," J. Chem. Phys. {\bf 3}, (to be published).

\bibitem{gallo99} K. Gallo and G. Assanto, ``All-optical diode based on second-harmonic generation in an asymmetric waveguide,'' \josab {\bf 16}(2), 267--269 (1999).

\bibitem{Masters98a} B. R. Masters, ``Three-dimensional microscopic tomographic imagings of the cataract in a human lens in vivo,'' \opex {\bf 3}(9), 332--338 (1998).

\bibitem{Oron03} D. Yelin,  D. Oron,  S. Thiberge,  E. Moses, and Y. Silberberg, ``Multiphoton plasmon-resonance microscopy,'' \opex {\bf 11}(12), 1385--1391 (2003).

\bibitem{samplefig}
B.~N.~Behnken, G.~Karunasiri, D.~R.~Chamberlin, P.~R.~Robrish, and J.~Faist,
``Real-time imaging using a 2.8~THz quantum cascade laser and uncooled infrared microbolometer camera,''
\ol \textbf{33}(5), 440--442 (2008).

\end{thebibliography}

%%%%%%%%%%%%%%%%%%%%%%%%%%  body  %%%%%%%%%%%%%%%%%%%%%%%%%%
\section{Introduction}
Adherence to the specifications listed in this template is essential for efficient review and publication of submissions. Since OSA does not routinely perform copyediting and typesetting for this journal, use of the template is critical to providing a consistent appearance. Proper reference format is especially important (see Section \ref{sec:refs}).

\section{Methods}


\section{Results}


\section{Discussion}

\subsection{Figures and tables}


\begin{figure}[h]
\centering\includegraphics[width=7cm]{test}
\caption{Sample caption (Ref. \cite{Oron03}, Fig. 2).}
\end{figure}

\end{document}
